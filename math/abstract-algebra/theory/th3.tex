\documentclass[10pt]{article}
\usepackage[utf8]{inputenc} 
\usepackage{amsfonts}
\usepackage{amsmath}
\usepackage{amssymb}
\usepackage[utf8]{inputenc}
\usepackage[bulgarian]{babel}
\usepackage{mathtools}

\newcommand*{\Z}{\mathbb{Z}}
\newcommand*{\N}{\mathbb{N}}
\newcommand*{\R}{\mathbb{R}}
\newcommand*{\Q}{\mathbb{Q}}
\newcommand{\triq}{\; \underline{\triangleleft} \;}

\begin{document}

\title{Теория 3}
\author{Валентин Стоянов}
\date{юни 2018}
\maketitle

\section*{Задача 1.}

\subsection*{Формулирайте теоремата за деление с частно и остатък за полиноми}
Нека $F$ е поле и $f, g \in F[x]: g \neq 0$. Тогава съществува единствена двойка полиноми $q, r \in F[x]$, такава че $f = qg + r$ и $deg(r) < deg(g)$.

\subsection*{Формулирайте схемата на Хорнер}
Нека $F[x]$ е поле и $f = a_0x^n + a_1x^{n-1} + \ldots + a_n, g = x - \alpha \in F[x]$. Нека $f = gq + r$, където $deg(r) < deg(g)$, т.е $r \in F$ и $q = b_0x^n-1+ \ldots + b_{n-1}$. Тогава следните формули са в сила:
\begin{center}
	$b_0 = a_0$,\\
	$b_1 = a_1 + \alpha b_0$,\\
	$\vdots$\\
	$b_{n-1} = a_{n-1} + \alpha b_{n-2}$,\\
	$r = a_n + \alpha b_{n-1}$
\end{center}

\subsection*{Какъв е видът на идеалите в пръстена от полиноми с коефициенти от дадено поле}
Главни идеали, породени от един елемент $f(x)$.

\subsection*{Колко най-много различни корени може да има ненулев полином от степен $n$ с коефициенти от дадена област}
$n$
\subsection*{Формулирайте принципа за сравняване на коефициентите на полиноми}

\section*{Задача 2.}

\subsection*{Напишете какво означава един полином да дели друг полином}
Нека $F$ е поле и $f, g \in F[x]: g \neq 0$. Ще казваме, че $g$ дели $f$($g \mid f$), ако съществува полином $q$, такъв че $f = qg$. Това означава, че остатъкът при деление на $f$ с $g$ е $0$.

\subsection*{Какво следва, ако даден полином дели произведението на два други полинома (всички полиноми са с коефициенти от дадено поле) и е взаимно прост с единия от тях}
Нека $F$ е поле и $f_1, f_2, g \in F[x]$. Ако $g \mid f_1f_2$ и НОД$(g, f_1) = 1$, то $g \mid f_2$.

\subsection*{Напишете определението за най-голям общ делител на два полинома}
Нека $F$ е поле и $f, g \in F[x]$ и поне един от тях е ненулев. Ще казваме, че един полином $d$ е най-голям общ делител(НОД) на $f$ и $g$ ($d = (f, g)$), ако $d$ удовлетворява следните условия:
\begin{itemize}
	\item $d \mid f, d \mid g$
	\item ако $d_1 \mid f, d_1 \mid g$, то $d_1 \mid d$.
\end{itemize}

\subsection*{Формулирайте тъждеството на Безу за два полинома}
Нека $F$ е поле и $f, g \in F[x]$ и $d = (f, g)$. Тогава съществуват полиноми $u$ и $v$, такива че $d = uf + vg$.

\subsection*{Напишете определението за най-малко общо кратно на два полинома}
Нека $F$ е поле и $f, g \in F[x]$ и поне един от тях е ненулев. Ще казваме, че един полином $k$ е най-малко общо кратно(НОК) на $f$ и $g$ ($k = (f, g)$), ако $d$ удовлетворява следните условия:
\begin{itemize}
	\item $f \mid k, g \mid k$
	\item ако $f \mid k_1, g \mid k_1$, то $k \mid k_1$.
\end{itemize}

\subsection*{Нека $f$ и $g$ са полиноми с коефициенти от дадено поле. Кой е пораждащият елемент на идеала $(f) + (g)$}
Нека $d = (f, g)$. Тогава пораждащият елемент на идеала $(f) + (g)$ е $d$.

\subsection*{Нека $f$ и $g$ са полиноми с коефициенти от дадено поле. Кой е пораждащият елемент на идеала $(f) \cap (g)$}

\subsection*{Напишете определението за неразложим полином над дадено поле}
Нека $F$ е поле и $f \in F[x] deg(f) > 0$. Ще казваме, че $f$ е неразложим над $F$, ако не може да се представи като произведение на два полинома с коефициенти от $F[x]$, чиито степени са по-ниски от тази на $f$.

\subsection*{Какво следва, ако един неразложим полином дели произведението на два други полинома (всички полиноми са с коефициенти от дадено поле)}
Нека $F$ е поле, $f_1, f_2, g \in F[x]$ и $g$ е неразложим полином. Ако $g \mid f_1f_2$, то $g \mid f_1$ или $g \mid f_2$.

\subsection*{Формулирайте теоремата за разлагане на полином на неразложими множители}
Нека $F$ е поле. Всеки неконстантен полином $f \in F[x]$ се разлага на произведение на неразложими над $F$ полиноми. Ако $f = p_1 \ldots p_k = q_1 \ldots q_s$ са две такива разлагания, то $k = s$ и, след евентуално преномериране на множителите, за всяко $i = 1,\ldots,k$ е изпълнено $p_i = a_iq_i, 0 \neq a_i \in F$.

\section*{Задача 3.}

\subsection*{Нека $F$ е поле и $f$ е неконстантен полином с коефициенти от $F$. Какъв е полиномът $f$, ако факторпръстенът $F[x]/(f)$ е поле}
Полиномът $f$ е неразложим над $F$.

\subsection*{Нека $F$ е поле и $f$ е неконстантен полином с коефициенти от $F$. Какъв вид пръстен е факторпръстенът $F[x]/(f)$, ако полиномът $f$ е неразложим}
Поле.

\subsection*{Напишете определението за поле на разлагане на полином над поле}
Нека $F$ е поле и $f \in F[x], deg(f) > 0$ и $L$ е разширение на полето $F$, което съдържа всички корени на полинома $f$. Сечението на всички подполета на $L$, съдържащи полето $F$ и всички корени на полинома $f$ ще наричаме поле на разлагане на $f$ над полето $F$.

\subsection*{Напишете формулите на Виет за полином от четвърта степен}
Нека $f = ax^4 + bx^3 + cx^2 + dx + e$ и $x_1,\ldots,x_4$ са корените на $f$. Тогава:
\begin{center}
	$x_1 + \ldots + x_4 = \frac{-b}{a}$,\\
	$x_1x_2 + x_1x_3 + x_1x_4 + x_2x_3 + x_2x_4 + x_3x_4 = \frac{c}{a}$,\\
	$x_1x_2x_3 + x_1x_2x_4 + x_1x_3x_4 + x_2x_3x_4 = \frac{-d}{a}$,\\
	$x_1x_2x_3x_4 = \frac{e}{a}$,\\
\end{center}

\subsection*{Напишете определението за $k$-кратен корен на полином}
Нека $F$ е поле, $charF = 0$ и $f \in F[x]$. $K$ е разширение на $F$ и $\alpha \in K$. Тогава $\alpha$ е $k$-кратен корен на $f$ точно когато $f(\alpha) = f'(\alpha) = \ldots = f^{(k-1)}(\alpha) = 0$ и $f^{(k)}(\alpha) \neq 0$.

\subsection*{Напишете необходимото и достатъчно условие полином с коефициенти от поле с характеристика нула да има $k$-кратен корен}
Нека $F$ е поле и $f \in F[x]$. $f$ има $k$-кратен корен точно когато има общ корен с производната си.

\section*{Задача 4.}

\subsection*{Формулирайте лемата за старшия едночлен за полиноми на много променливи}
Нека $A$ е област и $0 \neq f,g \in A[x_1,\ldots,x_n]$. Тогава старшият едночлен на полинома $fg$ е равен на произведението на старшите едночлени на $f$ и $g$.

\subsection*{Напишете определението за лексикографска наредба на едночлени на $n$ променливи}
Нека $A$ е област и $u = ax_1{}^{i_1} \ldots ax_n{}^{i_n}$ и $v = bx_1{}^{j_1} \ldots bx_n{}^{j_n}$ са два неподобни едночлена($= \neq a,b \in A$). Ще казваме, че едночленът $u$ е по-голям от едночлена $v$ и ще пишем $u > v$, ако съществува естествено число $k \leq n$, такова че $i_1 = j_1, \ldots , i_{k-1} = j_{k-1}$, но $i_k > j_k$.

\subsection*{Напишете определението за симетричен полином}
Нека $A$ е област и $f = f(x_1, \ldots ,x_n) \in A[x_1, \ldots ,x_n]$. Ще казваме, че $f$ е симетричен полином, ако за всяка пермитация $\sigma$ от симетричната група $S_n$ е изпълнено равенството $f(x_1, \ldots ,x_n) = f(x_{\sigma(1)}, \ldots ,x_{\sigma(n)})$.

\subsection*{Напишете $\sigma_2(x_1, x_2, x_3, x_4)$}
$x_1x_2 + x_1x_3 + x_1x_4 + \ldots + x_3x_4$

\subsection*{Формулирайте основната теорема за симетричните полиноми}
Нека $A$ е област и $f = f(x_1, \ldots ,x_n) \in A[x_1, \ldots ,x_n]$ е симетричен полином. Тогава съществува единствен полином $g$ на $n$ променливи с коефициенти от $A$, такъв че $f(x_1, \ldots ,x_n) = g(\sigma_1, \ldots, \sigma_n)$.
 
\subsection*{Напишете формулите на Нютон}
$S_k - \sigma_1S_{k-1} + \sigma_2S_{k-2} - \ldots + (-1)^{k-1}\sigma_{k-1}S_1 + (-1)^kk\sigma_k = 0$

\section*{Задача 5.}

\subsection*{Напишете определението за дискриминанта на полином}
Елемента $D(f) = a_0{}^{2n-2}\prod_{1 \leq i < j \leq n}^{}(\alpha_i - \alpha_j)^2$ (при $n > 1$) на полето $L$ ще наричаме дискриминанта на полинома $f$. Дискриминантата на полином от първа степен по определение е равна на 1.

\subsection*{Напишете формулата за дискриминанта на полином изразена чрез стойностите на производната на полинома за корените му}
$D(f) = a_0{}^{n-2}(-1)^{\frac{n(n - 1)}{2}}f'(\alpha_1) \ldots f'(\alpha_n)$.

\subsection*{Напишете определението за резултанта на два полинома}
Елемента $R(f,g) = a_0{}^sb_0{}^n\prod_{i=1}^{n}\prod_{j=1}^{s}(\alpha_i - \beta_j)$ на полето $L$ ще наричаме резултанта на полиномите $f$ и $g$.

\subsection*{Как се изразява резултантата на два полинома, ако знаем и корените на първия полином}
$R(f,g) = a_0{}^s\prod_{i=1}^{n}g(\alpha_i)$.

\subsection*{Как се изразява резултантата на два полинома, ако знаем и корените на втория полином}
$R(f,g) = b_0{}^n\prod_{i=1}^{s}g(\beta_i)$.

\subsection*{Как се изразява дискриминантата на даден полином чрез резултантата на полинома и производната му}
$R(f, f') = a_0(-1)^{\frac{n(n-1)}{2}}D(f)$.

\section*{Задача 6.}

\subsection*{Напишете определението за алгебрически затворено поле}
Ще казваме, че едно поле $F$ е алгебрически затворено, ако всеки неконстантен полином с коефициенти от $F$ има корен в $F$.

\subsection*{Формулирайте лемата на Гаус за полиноми с реални коефициенти}
Всеки неконстантен полином с реални коефициенти има поне един комплексен корен.

\subsection*{Формулирайте теоремата на Даламбер}
Полето на комплексните числа е алгебрически затворено.

\subsection*{Формулирайте основната теорема на алгебрата}

\subsection*{Какви могат да бъдат неразложимите полиноми с комплексни коефициенти}
\subsection*{Какви могат да бъдат неразложимите полиноми с реални коефициенти}


\section*{Задача 7.}

\subsection*{Напишете определението за примитивен полином}
Нека $f = a_0x^n + \ldots + a_n \in Z[x]$. Ще казваме, че $f$ е примитивен полином, ако НОД($a_0, \ldots, a_n$) е равен на $1$, т.е коефициентите му са взаимно прости.

\subsection*{Формулирайте лемата на Гаус за полиноми с цели коефициенти}
Произведение на два примитивни полинома също е примитивен полином.

\subsection*{Формулирайте редукционния критерий за неразложимост на полиноми с цели коефициенти}


\subsection*{Формулирайте критерия на Айзенщайн за неразложимост на полиноми с цели коефициенти}
Нека $f = a_0x^n + \ldots + a_n \in Z[x]$ и съществува просто число $p$, удовлетворяващо следните условия:
\begin{itemize}
	\item $p$ не дели $a_0$
	\item $p \mid a_1, \ldots, a_n$
	\item $p^2$ не дели $a_n$
\end{itemize}

\end{document}