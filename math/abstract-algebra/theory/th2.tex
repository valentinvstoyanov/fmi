\documentclass[10pt]{article}
\usepackage[utf8]{inputenc} 
\usepackage{amsfonts}
\usepackage{amsmath}
\usepackage{amssymb}
\usepackage[utf8]{inputenc}
\usepackage[bulgarian]{babel}
\usepackage{mathtools}

\newcommand*{\Z}{\mathbb{Z}}
\newcommand*{\N}{\mathbb{N}}
\newcommand*{\R}{\mathbb{R}}
\newcommand*{\Q}{\mathbb{Q}}
\newcommand{\triq}{\; \underline{\triangleleft} \;}

\begin{document}

\title{Теория 2}
\author{Валентин Стоянов}
\date{май 2018}
\maketitle

\section*{Задача 1.}

\subsection*{Напишете определението за комутативен пръстен}
Нека $(R, +, \times)$ е пръстен. $R$ се нарича комутативен пръстен, ако:\\
$\forall a, b \in R: \quad a \times b = b \times a$

\subsection*{Напишете определението за пръстен с единица}
Нека $(R, +, \times)$ е пръстен. $R$ е пръстен с единица, ако:\\
$\exists e \in R: \quad \forall a \in R: \quad e \times a = a \times e = a$

\subsection*{Напишете определението за област на цялост}

\subsection*{Напишете определението за делител на нулата в пръстен}
Нека $(R, +, \times)$ е пръстен. Елементът $a \in R, a \neq 0$ се нарича делител на нулата, ако $\exists b \in R, b \neq 0: \quad a \times b = 0$.

\subsection*{Напишете определението за тяло}
Нека $(R, +, \times)$ е пръстен с $1(\neq 0)$. $R$ се нарича тяло, ако:\\
$\forall a \in R, a \neq 0 \quad \exists a' \in R: \quad a \times a' = a' \times a = 1$

\subsection*{Напишете определението за поле}
Нека $(R, +, \times)$ е тяло. $R$ се нарича поле, ако:\\
$\forall a, b \in R: \quad a \times b = b \times a$

\subsection*{Напишете определението за подпръстен}
Нека $(R, +, \times)$ е пръстен и $H \subseteq R$. $H$ е подпръстен на $R$($H \leq R$), ако:\\
$\forall a, b \in H: \quad a \pm b \in H \land a \times b \in H$

\subsection*{Напишете определението за мултипликативната група на пръстен}

\section*{Задача 2.}

\subsection*{Напишете определението за характеристика на поле}
\subsection*{Какво число може да бъде характеристиката на едно поле}
\subsection*{Напишете определението за подполе}
\subsection*{Напишете определението за разширение на поле}
\subsection*{Напишете определението за просто поле}
\subsection*{С точност до изоморфизъм, кое поле може да бъде просто подполе на едно поле}

\section*{Задача 3.}

\subsection*{Напишете определението за ядро на хомоморфизъм на пръстени}
\subsection*{Напишете определението за образ на хомоморфизъм на пръстени}
\subsection*{Напишете определението за хомоморфизъм на пръстени}
\subsection*{Напишете определението за изоморфизъм на пръстени}

\section*{Задача 4.}
\subsection*{Напишете определението за ляв идеал на пръстен}
\subsection*{Напишете определението за десен идеал на пръстен}
\subsection*{Напишете определението за двустранен идеал на пръстен}
\subsection*{Напишете определението за сума на идеали}
\subsection*{Напишете определението за главен идеал, породен от елемент, в комутативен пръстен с единица}
\subsection*{Какъв е видът на идеалите в пръстена на целите числа $\Z$}
\subsection*{Как се дефинира операцията събиране във факторпръстен}
\subsection*{Как се дефинира операцията умножение във факторпръстен}
\subsection*{Формулирайте теоремата за хомоморфизмите за пръстени}

\section*{Задача 5.}
\subsection*{Докажете, че ако $P$ е поле, то $P$ няма нетривиални идеали (т.е. различни от ${0}$ и $P$}
\subsection*{Докажете, че ако един комутативен пръстен с единица P няма нетривиални идеали (т.e. различни от {0} и P), то P е поле}
\subsection*{Докажете, че всяко поле съдържа единствено просто подполе}

\end{document}